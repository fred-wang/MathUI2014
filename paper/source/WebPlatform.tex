\section{The Web Platform}

\subsection{Overview}

All the technologies presented in this section are based on Web standards and
should be supported by any Web rendering engine. We will particularly be
interested in Gecko which has the best native MathML support and is the core
of Firefox OS. We will present some of the improvements that have been made
by the Mozilla MathML team.

All these technologies should be usable in HTML documents. This obviously
includes Web pages but also EPUB ebooks, HTML mails, browser add-ons or
Firefox OS Web apps. For example, it is possible to receive and send emails
with mathematical equations using Thunderbird or Seamonkey's mail client
\cite{MathInEmails}.
In this paper, we will mainly
focus on Firefox OS Web apps but one should keep in mind that all the features
apply in other contexts too.

\subsection{Basic HTML5 Features}
\label{sec:basic}

The main language is HTML5, which allows to create pages with headers,
paragraphs, tables, hyperlinks etc The well-known CSS language is used to
apply specific style to HTML5 and is powerful enough to produce advanced
designs. Finally, DOM/Javascript provides a programming language and enables
interactive documents and complex user interfaces. New HTML5 elements
gives other possibilities . For example the document
{\tt pendulum-20131125} of \cite{MathInEbooks} uses the
the {\tt <video>} tag to insert some sequences of a physics lecture.

For scientific documents, we need two other features: creating graphs, schemas,
diagrams etc and writing mathematical formulas. For the former, simple PNG
images might be enough. However, Web rendering engines also support the SVG
language to let authors write scalable images using some simple drawing
primitives. Many programs are available to generate scientific schemas in SVG
formats. Mathematical equations can be viewed as an extension of text layout
and thus requires a good integration within HTML as done by
Gecko's native MathML. The document \href{http://fred-wang.github.io/MathUI2014/demos/1-mathml-in-html.html}{demos/1-mathml-in-html.html} shows how
various font properties apply to MathML text via CSS, the good alignment
of inline equations and its scalability.

One of the nice feature introduced some years ago in Gecko is the possibility
to integrate MathML equations inside SVG images via the {\tt <foreignObject>}
element. People can then create
scientific schemas with mathematical equations.
We will also use this property in section \ref{sec:advanced}
when we introduce {\tt <canvas>}.
See \href{http://fred-wang.github.io/MathUI2014/demos/2-mathml-in-svg.svg}{demos/2-mathml-in-svg.svg} for an example of a SVG schema with
MathML equations inside. Again, some authoring tools exist to help generating
such schemas.

\href{http://fred-wang.github.io/MathUI2014/demos/3-mathml-javascript.html}{demos/3-mathml-javascript.html} is a small example of an
interactive MathML formula. Javascript is used to allow the user to highlight
parts of a 3-dimensional determinant and understand each term of the
Sarrus' rule. Note that no particular Javascript API is needed, one only
modifies the MathML tree via the standard DOM interface and the rendering
is automatically updated.
This example is taken from \cite{MathInEbooks} which contains
many other examples of this type.

To conclude this review, we briefly mention classical Web features like
Unicode characters, complex text layout (e.g. ligatures),
right-to-left directionality, copy and paste, automatic line breaking and
accessibility to users with disabilities.
In general, all these features are well handled by Web languages and rendering
engines. The two first works well in Gecko's MathML
too and right-to-left MathML was also implemented four years ago
\cite{BugzillaLTR}.
For the fourth one, we have proposed a
\href{https://addons.mozilla.org/en-US/firefox/addon/mathml-copy/}{MathML Copy add-on} for Firefox as a temporary solution to
\cite{BugzillaClipboard}. MathML line breaking \cite{BugzillaLinebreaking}
is very important
for small screens but to our knowledge, no CSS-compatible implementations exist.
Finally, MathML accessibility is not available in Gecko yet although it is
provided by many other tools \cite{BugzillaAccessibility}. We expect to
bring improvements to these three last points in order to get a complete Web
platform for mathematical user interfaces.

\subsection{Styling of Mathematics}

As seen in section \ref{sec:basic}, one can use CSS the standard way to apply
specific styles to mathematical equations. However, perhaps the most important
style that scientists and publishers want is the rendering with a given
mathematical font. Gecko has always used TeX heuristics to position scripts,
fractions, roots etc but these rules do not necessarily work well for other
fonts than Knuth's Computer Modern. Moreover, in order to draw stretchy and
large operators the MathML code requires to pick and assembly specific glyphs
from math fonts. So far, Gecko only knows a limited set of Unicode constructions
and also has a
few private per-font tables. This means that without the appropriate
fonts installed, the quality of the MathML rendering could be very low.

In order to solve that problem, we have started to implement Microsoft's
OpenType MATH table \cite{OpenTypeMATHtable}, which is currently undergoing
standardization at the MPEG group.
We have started to use this table in order to provide a generic
support for MATH fonts. 
Note that MathML does not specify a precise way to render mathematical formulas
and leave the details to the implementers. Relying on the MATH table will give
more control to professional font designers.
In a nutshell, the main features are:

\begin{enumerate}

\item Some OpenType Tags to provide alternate form of glyphs. For example,
  we implemented the "ssty" feature to adjust the size of glyphs like primes
  when they are used as scripts.

\item Font parameters to define precisely the gaps, shifts, kerning etc of
  mathematical objects. Most of them are extensions to the TeX rules so it
  is easy to integrate them into our MathML rendering engine.

\item Font parameters specific to each glyph. At the moment, we only considered
  italic corrections.

\item A table to draw stretchy and large operators such as parenthesis,
  radicals or summation symbols. We have started to use that table for our
  operator stretching code.

\end{enumerate}

There are various mathematical fonts with an OpenType MATH table available and
most of them are distributed with a TeXLive distribution. Windows and Mac
systems have respectively Cambria Math and STIX available. We have also
experimented with PackageKit auto-installation on Linux. This means that
missing appropriate fonts will become very unlikely on Desktop. We also have
plans to make these fonts preinstalled on Firefox OS
\cite{BugzillaFirefoxOSFonts}. In any case, fallback
downloadable Web fonts can also be used.

The page \href{http://fred-wang.github.io/MathUI2014/demos/4-mathml-fonts.html}{demos/4-mathml-fonts.html} shows how to get various font style for
a document: Asana, TeX Gyre Bonum, TeX Gyre Pagella, TeX Gyre Termes,
Latin Modern, Neo Euler and XITS. At the moment, it will not work correctly
without a Nightly version of Gecko and some additional patches but we expect
that this work will be integrated into future Gecko releases.

\subsection{TeXZilla}
\label{sec:texzilla}

Mathematical formulas are complex and hence writing mathematics is difficult.
Some tools like WYSIWYG editors or handwriting recognition might help. One
input method easy to implement is the conversion from a simple syntax like
LaTeX into MathML. Even if there are tons of such converters, only a few of
them are implemented in Javascript. Moreover
the remaining ones try to tweak the output
to workaround browser limitations, are not standalone Javascript module usable
in add-ons and do not work with Unicode characters or right-to-left
mathematics. Hence we decided to write yet another converters to focus on
Mozilla's needs.

We thus wrote TeXZilla \cite{TeXZilla},
a LaTeX-to-MathML converter generated with the help of
\href{http://zaach.github.io/jison/docs/}{Jison}.
It relies on a LALR(1) grammar and on the {\tt unicode.xml} file of
\cite{XMLEntityDefinition}. It accepts arbitrary
Unicode input as well as right-to-left mathematics, something that is important
for the world-wide aspect of the Mozilla community and more specifically
Arabic mathematics.

The public API contains a {\tt toMathML} function to convert a LaTeX string
into a MathML DOM but also a {\tt toMathMLString} function when the DOM is
not available (e.g. in nodejs). There is also a convenient {\tt getTeXSource}
function to extract the LaTeX source from the MathML output and a
{\tt toImage} function to embed the MathML output inside a SVG image
(see section \ref{sec:advanced}).

TeXZilla can work in any CommonJS program, in Web pages, as a command line
program or as a Web server. We have submitted a \href{https://addons.mozilla.org/en-US/firefox/addon/texzilla/}{Firefox add-on}
that has been approved by the reviewers. We have also already integrated it into
various other Mozilla tools: in CKEditor for the Mozilla Developer Network,
in the Seamonkey/Thunderbird composers or in a Firefox OS webapp
(see section \ref{sec:texzillapp}).
Finally, a prototype {\tt <x-tex>} custom element is also available
(see section \ref{sec:advanced}).
We expect it will become an important library for future scientific
applications.

\subsection{Advanced HTML5 Features}
\label{sec:advanced}

The {\tt <canvas>} element has been introduced in HTML5 to draw graphics via
JavaScript. This element may be more convenient than SVG in situations where
you want to generate schemas dynamically. For example \href{http://fred-wang.github.io/MathUI2014/demos/5-canvas.html}{demos/5-canvas.html}
shows the typical graphs associated to a RLC circuit that are automatically
updated when the user changes the frequency of the current. It is also possible
to use WebGL \cite{WebGLspec},
opening the possibility to draw 3D scientific schemas. For example
\href{http://www.chemdoodle.com/}{chemdoodle}
relies on WebGL to provide 3D representation of chemical structures.

As for SVG, 2D/3D scientific schemas created via the {\tt <canvas>} might
require mathematical formulas. This is possible since we saw in section
\ref{sec:basic} that MathML can be
inserted in SVG via a {\tt <foreignObject>} and moreover SVG images can be
inserted in both the {\tt CanvasRenderingContext2D} and
{\tt WebGLRenderingContext}. One
can for example use TeXZilla's {\tt toImage} function to generate such SVG
images dynamically. For instance,
the page \href{http://fred-wang.github.io/MathUI2014/demos/6-mathml-in-webgl.html}{demos/6-mathml-in-webgl.html} shows an
example of a 3D schemas for the magnetic field of a circular current loop.
Again, see \cite{MathInEbooks} for more examples.

One of the recent HTML5 features that is currently being implemented in Web
rendering engines is Web Components \cite{WebComponents}.
The idea is to allow Web developers to
create their own custom HTML elements. These elements are made of the so-called
shadow tree, which is an ``internal'' representation from basic elements and CSS
style to which we attach some behaviors with DOM events and Javascript. This is
typically useful to create some reusable widgets. Mozilla's Brick project
is a collection of such elements and contains many demos \cite{Brick}.
Web Components are currently
not fully implemented natively in browsers but some polyfill libraries like
X-Tag allows to emulate their behavior \cite{XTag}.

Web components can obviously be very helpful to design user inteface for
sciencific Web app. For example, the specific interactive drawing of
\href{http://fred-wang.github.io/MathUI2014/demos/5-canvas.html}{demos/5-canvas.html} for RLC circuits could become a generic widget 
{\tt <x-graph>} to draw scientific graphs with some user interface to
configure the curve parameters. As a proof-of-concept, we have used the X-Tag
library to create a custom {\tt <x-tex>} tag. The behavior is simply to
automatically converts its LaTeX content into MathML using TeXZilla, see
\href{http://fred-wang.github.io/MathUI2014/demos/7-web-component.html}{demos/7-web-component.html}.
The Mozilla MathML team is following the developments in Web components and
plans to create more custom elements for sciencific web apps, so that Web
developers can reuse them.
