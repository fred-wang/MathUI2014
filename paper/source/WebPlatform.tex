\chapter{The Web Platform}

\section{Overview}

All the technologies presented in this chapter are based on Web standards and
should be supported by any Web rendering engines. However, for scientific
documents the best implementation is Gecko.

All these technologies should be usable in HTML documents. This obviously
includes Web pages but also EPUB ebooks, HTML mails, Browser add-ons, or
FirefoxOS Web apps. For example, it is possible to receive and send emails
with mathematical equations using Thunderbird or Seamonkey's mail client.
In this paper, we will mainly
focus on FirefoxOS Web apps but keep in mind that all the features apply in
other context too.

\section{Basic HTML5 Features}

The main language is HTML5, which allows to create pages with headers,
paragraphs, tables, hyperlinks etc The well-known CSS language is used to
apply specific style to HTML5 and is powerful enough to produce advanced
designs. Finally, DOM/Javascript provides a programming language and enables
interactive documents and complex user interface. New HTML5 elements
gives other possibilities . For example the document
{\tt pendulum-20131125} of the ``Mathematics in ebooks'' project uses the
the {\tt <video>} tag to insert some sequences of a physics lecture.

For scientific documents, we need two other features: creating graphs, schemas,
diagrams etc and writing mathematical formulas. For the former, simple PNG
images might be enough. However, Web rendering engines also support the SVG
language to let authors write scalable images using some simple drawing
primitive. Many programs are available to generate scientific schemas in SVG
formats. Mathematical equations can be viewed as an extension of text layout
and thus requires a good integration within HTML as done by
Gecko's native MathML. The document {\tt demos/1-mathml-in-html.html} shows how
various font properties apply to MathML text via CSS, the good alignment
of inline equations and its the scalability.

One of the nice feature introduced some years ago in Gecko is the possibility
to integrate MathML equation inside SVG images via the {\tt <foreignObject>}
element. People can then create
scientific schemas with mathematical equations.
We will also use this property later when we introduce {\tt <canvas>}.
See {\tt demos/2-mathml-in-svg.svg} for an example of a SVG schema with
MathML equations inside. LaTeXML 0.8 can generate
such schemas from the LaTeX commands of the TikZ package.

See also {\tt demos/3-mathml-javascript.html} for a small example of an
interactive MathML formula. Javascript is used to allow the user to highlight
each term of a 3-dimensional determinant and understand each term of the
Sarrus' rule. This is taken from the ``Mathematics in ebooks'' which contains
many other examples of this type.

\section{Styling of Mathematics}



\section{Canvas and WebGL}


\section{Web Components}


\section{TeXZilla}

