\section*{Introduction}

The Web has become an integral part of our daily life. Some of the reasons for
its success are its open nature and the way it enables anyone to get access to
knowledge and to create his own projects. The Mozilla community has worked since
the early days of the Web to guarantee openness, innovation and opportunity
\cite{MozillaMission}.

The Web was created at the CERN to share knowledge between researchers
\cite{WebCreation}.
But although it was invented by scientists, we still have not seen
the same positive impact on scientific practice. There are two main reasons that
could explain this situation and they are actually related each other.

The first one is a human problem. Researchers have kept teaching students
to write papers for publication in journals and to avoid sharing their detailed
results because of competition between academic groups. This means that most
scientists ignore
how to use tools to publish Web content and do not have the culture of openness
and collaboration. Mozilla Science Lab was launched last year to remedy that
problem and build educational resources, tools and prototypes for the research
community \cite{MozillaScienceLab}.

The second one is more technical. At the beginning of the Web, we had no tools
to write scientific documents that could effectively replace the traditional ones
for publication on papers. Moreover, the tools currently available are not used
during the early years of undergraduate and graduate courses.
In science, there is one extra technical problem:
we still lack a cross-compatible way to publish
mathematics on the Web despite the publication of the MathML standard in 1998
\cite{W3CMathHome}.
The Mozilla MathML Project \cite{MozillaMathML}
was launched in 1999 and in a few years, the team produced a good MathML
implementation in Gecko together with tools to publish mathematics on the Web.
MathML finally
became part of HTML5 thanks to Mozilla's effort \cite{MozillaMathMLHTML5},
but other Web rendering
engines still have limited support or even no support at all
\cite{MathMLForgesOn}. This means that
scientists either stay outside the Web (e.g. exchange only PDF documents) or
rely on some workarounds (PNG images, CSS stylesheet, Javascript polyfills)
to publish mathematical content on the Web, with their inherent limitations and
issues.

In recent years, the mobile market has grown considerably and more and more
people are using mobile devices to access the Web. Mozilla has been working
on Firefox OS, an open-source operating system for these mobile devices that
relies exclusively on open standards and in particular Web technologies
\cite{MozillaFirefoxOS}. Thanks to
Gecko's good support for MathML and HTML5 in general as well as
Mozilla's long experience with community involvement, we now have the opportunity
to build a family of scientific Web applications for Firefox OS devices.
Some of the early prototypes created by the Mozilla MathML team are presented
in this paper.

In a first part, we will review the Web platforms and focus on how the
technologies can be used for science. We will present the classical features
as well as more recent improvements that have been integrated in Mozilla
projects recently such as the Open Type MATH table, WebGL, Web Components or
TeXZilla. Some of this work has been made during the crowdfunding project
``Mathematics in ebooks'' which has also resulted in the creation of collection
of scientific documents using advanced Web technologies \cite{MathInEbooks}.

In a second
part, we will study how to use these technologies to write Firefox OS Web apps
for Science and introduce a few of the tools recently developed by the Mozilla
MathML team. TODO: list the apps!

The authors have written this paper using collaboration tools like GitHub and
all the sources, programs and tools presented here are open. Because 
\href{http://fred-wang.github.io/MathUI2014/paper/output/MathUI2014-MozillaMathML.pdf}{PDF format}
has been requested for submission to the MathUI workshop, we had to provide
\href{http://fred-wang.github.io/MathUI2014/demos/}{our demos} separately instead of integrating them directly in the document.
However, we invite the reader to test the demos in a Gecko browser. A 
\href{http://fred-wang.github.io/MathUI2014/paper/output/}{Web version} of that document is also available for online reading.
