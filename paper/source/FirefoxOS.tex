\chapter{Firefox OS}

\section{Overview}

Firefox OS is the open-source operating system targeting mobile devices being
developed by Mozilla and based on Web technologies. All the apps in Firefox OS
are Web pages and to create a app from a previous Web page you only need a manifest
file that store some metadatas about the app and a figure to be the icon. The
documents {\tt demos/app/manifest.webapp} and {\tt demos/app/icon.png} show how
create a app for {\tt demos/app/index.html}.

In the next section we give more details of what we call ``Math Suite'' and
after that we present some of the demos wrote by Mozilla MathML community.

\section{Math Suite}

Like an office suite is a collection of softwares intended to be used by
knowledge workers an Math Suite is a collection of softwares intended to be used
by someone that make intensive use of math and should have tools like markup
converters, WYSIWYG editor, handwriting recognition and more.

Many of the applications in an Math Suite share a core set of functions and have
this functions available in a open source Javascript library will make easy to
develop new applications in the same way as Basic Linear Algebra Subprograms
(BLAS), a specification for common operations using scalar and vectors, wrote in
1979 are still used as a building block in higher-level math programming
languages and libraries.

The authors of this work start writing this Javascript libraries
and some demos that use it.

\section{Math Cheat Sheet}

This is the most basic app that someone can write since it
is a collection of common K-12 equations available as a app.

The equations are organized in sections and a table of contents is provided to
help jumping from one section to another. Nice features that can be add in the
future include:
\begin{itemize}
  \item link to resource like Wikipedia related to the equation,
  \item search bar to quickly find the desired equation,
  \item customization of equations, \ldots
\end{itemize}

\section{TeXZilla App}

This is a note block app for math that take (La)TeX as input.
