\section*{Conclusion}

In this paper, we have reviewed the HTML5 features and how they could be used
to write mathematical user interfaces. In particular, good MathML support is
essential to write mathematics in Web apps in a way that is compatible with the
classical HTML, CSS, Javascript/DOM and SVG features. We have also presented
some of the improvements to mathematical styling in Gecko based on OpenType
MATH fonts. We then gave an overview of TeXZilla, a standalone LaTeX-to-MathML
converter compatible with Unicode that can easily be integrated in Web apps.
Finally, we studied advanced HTML5 features like canvas, WebGL and Web
Components and mentioned how they could be used to create 3D-schemas, animations
or complex user interfaces.
These are suitable to share knowledge in science and help
explaining complex concepts more easily, in an interactive way.

In a second part, we explained how these HTML5 features can be used to create
Firefox OS Web apps and we provided concrete use cases like a note-taking math
app, a math cheat sheet or an interactive app for algebraic manipulations.
The idea is to rely on Firefox OS to build a platform for science, not only to
consult knowledge but also to generate new knowledge as shown in some of the
demos presented in this papers. Although, these apps are still prototypes we
expect we will improve them in the future and create a real math suite for
mobile platforms. We expect that new contributors could get involved and that
the original ideas shared at the MathUI workshop could in turn become real
Firefox OS apps.
